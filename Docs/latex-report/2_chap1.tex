\chapter*{Введение}
\addcontentsline{toc}{chapter}{Введение}
\label{ch:chap1}



В современном мире мультикоптерные летательные аппараты, 
в частности квадрокоптеры, находят применение во всем большем количестве 
сфер. Они используются для сбора информации об окружающей 
среде, мониторинга территорий, доставки грузов, расследований
и даже съемок кино. 
Широкое распространение квадрокоптеров обусловлено 
их маневренностью, относительной простотой конструкции и 
возможностью выполнения задач в условиях, недоступных 
для других типов летательных аппаратов. Также растет и доступность многоротерных 
летательных аппаратов за счет удешевления компонентов.

Квадрокоптер представляет собой сложную механическую систему, состоящую из 
четырех электродвигателей расположенными симметрично по углам рамы. 
Винты, раскручиваемые электродвигателями, создают тягу, которая поднимает квадрокоптер в воздух, 
позволяя ему удерживаться и маневрировать. В современных квадрокоптерах 
используются бесколлекторные электродвигатели, получившие широкое 
распространение в последние десятилетия за счет их эффективности, 
простой и надежной конструкции. Также одними из наиболее важных 
компонентов является микроконтроллер, называемый полетным контроллером - 
он отвечает за управление напряжением на моторах. Помимо этого квадрокоптеры 
оснащены драйверами двигателей, средствами коммуникации и различными датчиками: 
акселерометрами, gps-датчиками, камерами и в некоторых случаях лидарами.

Для создания законов управления и компьютерного моделирования необходимо разработать
математическую модель динамики летательного аппарата. Поведение мультироторного 
устройства описывается нелинейными дифференциальными уравнениями, их 
количество может разнится в зависимости от количества электродвигателей 
и учета внешних возмущений. Математическая модель может быть синтезирована с использованием 
уравнений Эйлера-Лагранжа или Ньютона-Эйлера, которые учитывают поступательное и 
вращательное движение аппарата.  Квадрокоптер обладает шестью степенями свободы:
тремя поступательными - движение вдоль осей X, Y, Z, и тремя вращательными - углы крена, тангажа, рыскания.
Для упрощения задачи синтеза законов управления  используется 
линеаризованная модель квадрокоптера, полученная путем аппроксимации 
нелинейных уравнений около состояния равновесия, что позволяет применять 
классические методы теории управления для анализа системы и синтеза законов управления.


Для выполнения работы в автономном режиме - квадрокоптер должен обладать 
робастной системой управления, которая будет не восприимчива к значительным 
возмущениям и изменениям внешней среды. Для этого необходимо 
использовать алгоритмы с робастными модификациями или адаптивными законами.


Задача слежения за траекторией является базовой для 
автономного квадрокоптера. Суть задачи заключается в  
способности аппарата точно следовать по заранее заданной 
траектории, что особенно важно в условиях отсутствия связи и полном отсутсвии управления человеком. 
Слежение за траекторией требует системы управления обеспечивающей стабилмзацию устройства, движение по траектории и усточивости по отношению к внешним возмущениям.


В настоящей работе рассматривается исследование и разработка 
алгоритмов управления квадрокоптером для решения задачи слежения 
за траекторией. В рамках работы планируется решить следующие задачи:

\begin{enumerate}
    \item Cоздание математической модели квадрокоптера, учитывающей его динамику и возможные 
    внешние возмущения.
    \item Синтез алгоритмов управления обеспечивающих слежение за заданной траекторией
    \item Компьютерное моделирование в среде MATLAB/Simulink 
    для оценки эффективности разработанных алгоритмов
    \item Экспериментальная проверка алгоритмов на реальном 
    оборудовании, что позволит оценить их применимость в реальных условиях
\end{enumerate}


Результаты работы могут быть использованы для улучшения систем управления 
квадрокоптерами в различных приложениях.
  




\endinput
