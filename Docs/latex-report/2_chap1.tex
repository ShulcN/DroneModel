\chapter*{Введение}
\addcontentsline{toc}{chapter}{Введение}
\label{ch:chap1}



В современном мире мультикоптерные летательные аппараты, 
в частности квадрокоптеры, находят применение во всем большем количестве 
сфер. Они используются для сбора информации об окружающей 
среде, мониторинга территорий, доставки грузов, расследований
и даже съемок кино. 
Широкое распространение квадрокоптеров обусловлено 
их маневренностью, относительной простотой конструкции и 
возможностью выполнения задач в условиях, недоступных 
для других типов летательных аппаратов. Также растет и доступность многоротерных 
летательных аппаротов за счет удешевления бесколлекторных моторов и микроконтроллеров. 


Квадрокоптер представляет собой сложную механическую систему, состоящую из 
четырех электродвигателей расположенными симметрично по углам рамы. 
Электродвигатели вращающие винты создают тягу, которая поднимает квадрокоптер
в воздух и позволяет ему летать. 
Поведение мультироторного устройства описывается нелинейными дифференциальными 
уравнениями. 
Алгоритмы управления квадрокоптером должны быть робастными для устойчивости 
к внешним возмущениям, таким как ветер и дождь.


Одной из основных сложностей при создании системы управления 
квадрокоптером является учет нелинейности его динамики и 6 
степеней свободы устройства. Задача слежения за траекторией является базовой для 
автономного квадрокоптера. Она предполагает 
способность аппарата точно следовать по заранее заданной 
траектории, что особенно важно в таких приложениях, как 
автономная навигация, картографирование или выполнение задач
в сложных условиях. 




В настоящей работе рассматривается исследование и разработка 
алгоритмов управления квадрокоптером для решения задачи слежения
за траекторией. В рамках работы планируется 
разработать математическую модель квадрокоптера, 
синтезировать алгоритмы управления, провести их моделирование и 
экспериментальную проверку.
  




\endinput
