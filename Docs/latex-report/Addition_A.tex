\chapter{Приложение А}

Листинг кода основной программы STM32F722.

\begin{lstlisting}[language=C++]
#include "main.h"
#include "usb_device.h"

/* Private includes ----------------------------------------------------------*/
/* USER CODE BEGIN Includes */
#include <stdio.h>
#include "usbd_cdc_if.h"
#include <icm4268.h>
#ifndef MATH
#include <math.h>
#define MATH
#endif
//#include "ICM42688.h"
/* USER CODE END Includes */

/* Private typedef -----------------------------------------------------------*/
/* USER CODE BEGIN PTD */

/* USER CODE END PTD */

/* Private define ------------------------------------------------------------*/
/* USER CODE BEGIN PD */
typedef enum {
    DRONE_STATE_IDLE = 0,
    DRONE_STATE_TEST_CYCLE,
    DRONE_STATE_EMERGENCY_STOP,
    DRONE_STATE_LANDING,
    DRONE_STATE_LQR,
    DRONE_STATE_MOVE_FORWARD,
    DRONE_STATE_HOVER,
    DRONE_TEST_DIRS,
} DroneState;
volatile DroneState currentState = DRONE_STATE_IDLE;
volatile DroneState requestedState = DRONE_STATE_IDLE;
volatile uint32_t stateStartTime = 0;
volatile uint32_t landingStartAltitude = 0;
volatile uint8_t landingInProgress = 0;
/* USER CODE END PD */

/* Private macro -------------------------------------------------------------*/
/* USER CODE BEGIN PM */
#define RX_BUF_SIZE  16
static uint8_t rx_byte;
static char cmd_buf[RX_BUF_SIZE];
static uint8_t cmd_idx = 0;
static uint8_t mode = 0;
uint8_t rx_buf[5];
volatile uint8_t rx_idx = 0;
GPIO_TypeDef* ICM_CS_GPIO_Port = GPIOB;
uint16_t      ICM_CS_Pin       = GPIO_PIN_2;
float angles_rad[3], angles_deg[3], gyro_rad_s[3], angles_speed[3], angles_rad_prev[3];
int16_t accel_raw[3], gyro_raw[3];
float filtered_accel[3] = {0};
float filtered_gyro[3] = {0};
float dt = 0.01f;
float pos[3], prev_pos[3], pos_req[3], speed[3];
float k1=10,k2=5,k3=10,k4=5,k5=10,k6=5,k7=31623,k8=251,k9=31623,k10=251,k11=0,k12=0;
int requested_speed = 1220;
float m = 0.65, Ix=0.0021, Iy=0.0021, Iz=0.0043;
float kf=0.0000158, k_1_2_l=3.33, kf_inv=63463.8;
uint16_t motorPWM[4];
const float scale = 0.15f;
const float min_pwm = 1140.0f;
const float max_pwm = 1500.0f;
float bias[3];
uint32_t timer_lqr = 0;
/* USER CODE END PM */

/* Private variables ---------------------------------------------------------*/

SPI_HandleTypeDef hspi1;

TIM_HandleTypeDef htim2;
TIM_HandleTypeDef htim3;
TIM_HandleTypeDef htim4;
DMA_HandleTypeDef hdma_tim2_ch1;
DMA_HandleTypeDef hdma_tim2_ch2_ch4;
DMA_HandleTypeDef hdma_tim3_ch1_trig;
DMA_HandleTypeDef hdma_tim4_ch1;

UART_HandleTypeDef huart3;

/* USER CODE BEGIN PV */

/* USER CODE END PV */

/* Private function prototypes -----------------------------------------------*/
void SystemClock_Config(void);
static void MPU_Config(void);
static void MX_GPIO_Init(void);
static void MX_DMA_Init(void);
static void MX_TIM2_Init(void);
static void MX_TIM3_Init(void);
static void MX_TIM4_Init(void);
static void MX_SPI1_Init(void);
static void MX_USART3_UART_Init(void);
/* USER CODE BEGIN PFP */
/* USER CODE END PFP */

/* Private user code ---------------------------------------------------------*/
/* USER CODE BEGIN 0 */
void USB_CDC_RxHandler(uint8_t*, uint32_t);
int _write(int file, char *ptr, int len) {
    CDC_Transmit_FS((uint8_t*) ptr, len); return len;
}
void SetMotorOutputs(uint16_t ch1, uint16_t ch2, uint16_t ch3, uint16_t ch4)
{
    __HAL_TIM_SET_COMPARE(&htim2, TIM_CHANNEL_1, ch1);
    __HAL_TIM_SET_COMPARE(&htim2, TIM_CHANNEL_2, ch2);
    __HAL_TIM_SET_COMPARE(&htim3, TIM_CHANNEL_1, ch3);
    __HAL_TIM_SET_COMPARE(&htim4, TIM_CHANNEL_1, ch4);
}
void ArmESCs(void)
{
    SetMotorOutputs(1000, 1000, 1000, 1000);
    HAL_Delay(2000);
}
void testCycle(void){
    for (uint16_t dc = 1000; dc <= 1220; dc += 2) {
        SetMotorOutputs(dc, dc, dc, dc);
        HAL_Delay(80);
    }
    for (uint16_t dc = 1220; dc > 1000; dc -= 2) {
        SetMotorOutputs(dc, dc, dc, dc);
        HAL_Delay(80);
    }
    //SetMotorOutputs(1000, 1000, 1000, 1000);
}
void testMotorsDirs(void){
    SetMotorOutputs(1150, 1000, 1000, 1000);
    HAL_Delay(2500);
    SetMotorOutputs(1000, 1000, 1000, 1000);
    ///////////
    SetMotorOutputs(1000, 1150, 1000, 1000);
    HAL_Delay(2500);
    SetMotorOutputs(1000, 1000, 1000, 1000);
    /////////
    SetMotorOutputs(1000, 1000, 1150, 1000);
    HAL_Delay(2500);
    SetMotorOutputs(1000, 1000, 1000, 1000);
    ////////
    SetMotorOutputs(1000, 1000, 1000, 1150);
    HAL_Delay(2500);
    SetMotorOutputs(1000, 1000, 1000, 1000);
}
void stopMotors(void){
    //SetMotorOutputs(dc, dc, dc, dc);
    SetMotorOutputs(1000, 1000, 1000, 1000);
}
void performLanding(void) {
    static uint32_t landingTimer = 0;
    static uint16_t currentThrottle = 0;

    if (!landingInProgress) {
    landingInProgress = 1;
    landingTimer = HAL_GetTick();
    currentThrottle = 1200;
    printf("LANDING STARTED\r\n");
    }

    uint32_t elapsedTime = HAL_GetTick() - landingTimer;
    if (elapsedTime < 5000) {
    uint16_t newThrottle = 1000 + (uint16_t)((float)(currentThrottle - 1000) * (1.0f - (float)elapsedTime / 5000.0f));
    SetMotorOutputs(newThrottle, newThrottle, newThrottle, newThrottle);
    } else {
    SetMotorOutputs(1000, 1000, 1000, 1000);
    landingInProgress = 0;
    currentState = DRONE_STATE_IDLE;
    printf("LANDING COMPLETE\r\n");
    }
}

void performTestCycle(void) {
    static uint16_t currentDC = 1000;
    static uint8_t increasing = 1;
    static uint32_t lastUpdate = 0;

    if (requestedState != currentState) {
    return;
    }

    uint32_t now = HAL_GetTick();
    if (now - lastUpdate >= 80) {
    lastUpdate = now;

    if (increasing) {
        currentDC += 2;
        if (currentDC >= requested_speed)
        increasing = 0;
    } else {
        currentDC -= 2;
        if (currentDC <= 1020) {
        increasing = 1;
        currentState = DRONE_STATE_IDLE;
        requestedState = DRONE_STATE_IDLE;
        SetMotorOutputs(1000, 1000, 1000, 1000);
        printf("TEST CYCLE COMPLETE\r\n");
        return;
        }
    }

    SetMotorOutputs(currentDC, currentDC, currentDC, currentDC);
    }
}


void performLQRStable(void) {
    float LQR_SPEED[4] = {0, 0, 0, 0};

    if (requestedState != currentState) {
    return;
    }
    uint32_t now = HAL_GetTick();
    if (timer_lqr<now){
        float nu[6];
        nu[0] = -k1*(pos[0]-pos_req[0])-k2*speed[0];
        nu[1] = -k3*(pos[1]-pos_req[1])-k4*speed[1];
        nu[2] = -k5*(pos[2]-pos_req[2])-k6*speed[2];

        nu[5] = -k11*angles_rad[2]-k12*angles_speed[2];
        //uint32_t now = HAL_GetTick();
        float bar_phi = atan2f(nu[0], (nu[2]+9.81f));
        float bar_theta = atan2f(nu[1], (nu[2]+9.81f));

        nu[3] = -k7*(angles_rad[0]-bar_phi)-k8*angles_speed[0];
        nu[4] = -k9*(angles_rad[1]-bar_theta)-k10*angles_speed[1];

        float U[4];
        U[0] = m*(nu[2]+9.81)/(cosf(bar_phi)*cosf(bar_theta));
        U[1] = Ix*nu[3]-(Iy-Iz)*angles_speed[1]*angles_speed[2];
        U[2] = Iy*nu[4]-(Ix-Iz)*angles_speed[0]*angles_speed[2];
        U[3] = Iz*nu[5]-(Ix-Iy)*angles_speed[0]*angles_speed[1];
        float temps[4];
        temps[0] = kf_inv*(U[0]-k_1_2_l*U[2])-U[3];
        temps[1] = kf_inv*(U[0]-k_1_2_l*U[1])+U[3];
        temps[2] = kf_inv*(U[0]+k_1_2_l*U[2])-U[3];
        temps[3] = kf_inv*(U[0]+k_1_2_l*U[1])+U[3];
        printf("WORING\r\n");
//	  for (int i = 0; i < 4; i++) {
//		if (temps[i]<0){
//			printf("ERROR ZERO CROSSING IN SQRT\r\n");
//			return;
//		}
//	  }
        LQR_SPEED[0] = 0.5*sqrtf(temps[0]);
        LQR_SPEED[1] = 0.5*sqrtf(temps[1]);
        LQR_SPEED[2] = 0.5*sqrtf(temps[2]);
        LQR_SPEED[3] = 0.5*sqrtf(temps[3]);
        for (int i = 0; i < 4; i++) {
            float pwm = LQR_SPEED[i] * scale + min_pwm;

            pwm = fmaxf(pwm, min_pwm);
            pwm = fminf(pwm, max_pwm);

            motorPWM[i] = (uint16_t)pwm;
        }
        printf("motors: %d, %d, %d, %d\r\n", motorPWM[0], motorPWM[1],motorPWM[2],motorPWM[3]);

        SetMotorOutputs(motorPWM[3],motorPWM[2],motorPWM[0], motorPWM[1]);
    }

}

void handleDroneState(void) {
    if (requestedState != currentState) {
    switch (requestedState) {
        case DRONE_STATE_EMERGENCY_STOP:
        SetMotorOutputs(1000, 1000, 1000, 1000);
        printf("EMERGENCY STOP ACTIVATED\r\n");
        break;

        case DRONE_STATE_LANDING:
        landingInProgress = 0;
        break;

        case DRONE_STATE_TEST_CYCLE:
        printf("TEST CYCLE STARTED\r\n");
        break;

        case DRONE_STATE_LQR:
            timer_lqr = HAL_GetTick()+2500;
            printf("STARTED LQR\r\n");
            break;

        default:
        break;
    }

    currentState = requestedState;
    stateStartTime = HAL_GetTick();
    }

    switch (currentState) {
    case DRONE_STATE_IDLE:
        break;

    case DRONE_STATE_TEST_CYCLE:
        performTestCycle();
        break;

    case DRONE_STATE_LANDING:
        performLanding();
        break;

    case DRONE_STATE_EMERGENCY_STOP:
        SetMotorOutputs(1000, 1000, 1000, 1000);
        break;

    case DRONE_STATE_LQR:
        performLQRStable();
        break;

    default:
        break;
    }
}

/* USER CODE END 0 */

/**
    * @brief  The application entry point.
    * @retval int
    */
int main(void)
{

    /* USER CODE BEGIN 1 */

    /* USER CODE END 1 */

    /* MPU Configuration--------------------------------------------------------*/
    MPU_Config();

    /* MCU Configuration--------------------------------------------------------*/

    /* Reset of all peripherals, Initializes the Flash interface and the Systick. */
    HAL_Init();

    /* USER CODE BEGIN Init */

    /* USER CODE END Init */

    /* Configure the system clock */
    SystemClock_Config();

    /* USER CODE BEGIN SysInit */

    /* USER CODE END SysInit */

    /* Initialize all configured peripherals */
    MX_GPIO_Init();
    MX_DMA_Init();
    MX_USB_DEVICE_Init();
    MX_TIM2_Init();
    MX_TIM3_Init();
    MX_TIM4_Init();
    MX_SPI1_Init();
    MX_USART3_UART_Init();
    /* USER CODE BEGIN 2 */
    HAL_TIM_PWM_Start(&htim2, TIM_CHANNEL_1);
    HAL_TIM_PWM_Start(&htim2, TIM_CHANNEL_2);
    HAL_TIM_PWM_Start(&htim3, TIM_CHANNEL_1);
    HAL_TIM_PWM_Start(&htim4, TIM_CHANNEL_1);
    ArmESCs();
    //TestSPI();
    if (ICM_Init() != HAL_OK) {
        HAL_Delay(1000);
        printf("ERROR INIT GYRO\r\n");
    }
    else {
        ICM_CalibrateGyro(bias);
        printf("Calibration complete: bias_x=%.2f, bias_y=%.2f, bias_z=%.2f deg/s\r\n",
                bias[0], bias[1], bias[2]);
    }
    //testCycle();
    /* USER CODE END 2 */

    /* Infinite loop */
    /* USER CODE BEGIN WHILE */
    //HAL_UART_Receive_IT(&huart3, &rx_buf, 1);
    //HAL_Delay(12000);
    //testMotorsDirs();
    printf("Begin main loop\r\n");
    HAL_Delay(300);
    HAL_UART_Receive_IT(&huart3, &rx_byte, 1);

    uint32_t last_orientation_time = HAL_GetTick()-10;
    while (1)
    {
        HAL_Delay(10);
        handleDroneState();

        uint32_t now = HAL_GetTick();
        if (now - last_orientation_time >= 10) {
            if (ICM_UpdateEulerAngles() == HAL_OK) {
                memcpy(angles_rad, angles_rad_prev, 3);
                ICM_GetEulerAngles(angles_rad);
                    printf("ORIENT: %.2f, %.2f, %.2f\r\n", angles_rad[0], angles_rad[1], angles_rad[2]);
            } else {
                printf("ERROR ANGLES\r\n");
            }
            last_orientation_time = now;
        }

    /* USER CODE END WHILE */

    /* USER CODE BEGIN 3 */

    }
    /* USER CODE END 3 */
}

/**
    * @brief System Clock Configuration
    * @retval None
    */
void SystemClock_Config(void)
{
    RCC_OscInitTypeDef RCC_OscInitStruct = {0};
    RCC_ClkInitTypeDef RCC_ClkInitStruct = {0};

    /** Configure the main internal regulator output voltage
    */
    __HAL_RCC_PWR_CLK_ENABLE();
    __HAL_PWR_VOLTAGESCALING_CONFIG(PWR_REGULATOR_VOLTAGE_SCALE1);

    /** Initializes the RCC Oscillators according to the specified parameters
    * in the RCC_OscInitTypeDef structure.
    */
    RCC_OscInitStruct.OscillatorType = RCC_OSCILLATORTYPE_HSE;
    RCC_OscInitStruct.HSEState = RCC_HSE_ON;
    RCC_OscInitStruct.PLL.PLLState = RCC_PLL_ON;
    RCC_OscInitStruct.PLL.PLLSource = RCC_PLLSOURCE_HSE;
    RCC_OscInitStruct.PLL.PLLM = 8;
    RCC_OscInitStruct.PLL.PLLN = 432;
    RCC_OscInitStruct.PLL.PLLP = RCC_PLLP_DIV2;
    RCC_OscInitStruct.PLL.PLLQ = 9;
    if (HAL_RCC_OscConfig(&RCC_OscInitStruct) != HAL_OK)
    {
    Error_Handler();
    }

    /** Activate the Over-Drive mode
    */
    if (HAL_PWREx_EnableOverDrive() != HAL_OK)
    {
    Error_Handler();
    }

    /** Initializes the CPU, AHB and APB buses clocks
    */
    RCC_ClkInitStruct.ClockType = RCC_CLOCKTYPE_HCLK|RCC_CLOCKTYPE_SYSCLK
                                |RCC_CLOCKTYPE_PCLK1|RCC_CLOCKTYPE_PCLK2;
    RCC_ClkInitStruct.SYSCLKSource = RCC_SYSCLKSOURCE_PLLCLK;
    RCC_ClkInitStruct.AHBCLKDivider = RCC_SYSCLK_DIV1;
    RCC_ClkInitStruct.APB1CLKDivider = RCC_HCLK_DIV4;
    RCC_ClkInitStruct.APB2CLKDivider = RCC_HCLK_DIV2;

    if (HAL_RCC_ClockConfig(&RCC_ClkInitStruct, FLASH_LATENCY_7) != HAL_OK)
    {
    Error_Handler();
    }
}

/**
    * @brief SPI1 Initialization Function
    * @param None
    * @retval None
    */
static void MX_SPI1_Init(void)
{

    /* USER CODE BEGIN SPI1_Init 0 */

    /* USER CODE END SPI1_Init 0 */

    /* USER CODE BEGIN SPI1_Init 1 */

    /* USER CODE END SPI1_Init 1 */
    /* SPI1 parameter configuration*/
    hspi1.Instance = SPI1;
    hspi1.Init.Mode = SPI_MODE_MASTER;
    hspi1.Init.Direction = SPI_DIRECTION_2LINES;
    hspi1.Init.DataSize = SPI_DATASIZE_8BIT;
    hspi1.Init.CLKPolarity = SPI_POLARITY_LOW;
    hspi1.Init.CLKPhase = SPI_PHASE_1EDGE;
    hspi1.Init.NSS = SPI_NSS_SOFT;
    hspi1.Init.BaudRatePrescaler = SPI_BAUDRATEPRESCALER_32;
    hspi1.Init.FirstBit = SPI_FIRSTBIT_MSB;
    hspi1.Init.TIMode = SPI_TIMODE_DISABLE;
    hspi1.Init.CRCCalculation = SPI_CRCCALCULATION_DISABLE;
    hspi1.Init.CRCPolynomial = 7;
    hspi1.Init.CRCLength = SPI_CRC_LENGTH_DATASIZE;
    hspi1.Init.NSSPMode = SPI_NSS_PULSE_ENABLE;
    if (HAL_SPI_Init(&hspi1) != HAL_OK)
    {
    Error_Handler();
    }
    /* USER CODE BEGIN SPI1_Init 2 */

    /* USER CODE END SPI1_Init 2 */

}

/**
    * @brief TIM2 Initialization Function
    * @param None
    * @retval None
    */
static void MX_TIM2_Init(void)
{

    /* USER CODE BEGIN TIM2_Init 0 */

    /* USER CODE END TIM2_Init 0 */

    TIM_ClockConfigTypeDef sClockSourceConfig = {0};
    TIM_MasterConfigTypeDef sMasterConfig = {0};
    TIM_OC_InitTypeDef sConfigOC = {0};

    /* USER CODE BEGIN TIM2_Init 1 */

    /* USER CODE END TIM2_Init 1 */
    htim2.Instance = TIM2;
    htim2.Init.Prescaler = 107;
    htim2.Init.CounterMode = TIM_COUNTERMODE_UP;
    htim2.Init.Period = 2082;
    htim2.Init.ClockDivision = TIM_CLOCKDIVISION_DIV1;
    htim2.Init.AutoReloadPreload = TIM_AUTORELOAD_PRELOAD_DISABLE;
    if (HAL_TIM_Base_Init(&htim2) != HAL_OK)
    {
    Error_Handler();
    }
    sClockSourceConfig.ClockSource = TIM_CLOCKSOURCE_INTERNAL;
    if (HAL_TIM_ConfigClockSource(&htim2, &sClockSourceConfig) != HAL_OK)
    {
    Error_Handler();
    }
    if (HAL_TIM_PWM_Init(&htim2) != HAL_OK)
    {
    Error_Handler();
    }
    sMasterConfig.MasterOutputTrigger = TIM_TRGO_RESET;
    sMasterConfig.MasterSlaveMode = TIM_MASTERSLAVEMODE_DISABLE;
    if (HAL_TIMEx_MasterConfigSynchronization(&htim2, &sMasterConfig) != HAL_OK)
    {
    Error_Handler();
    }
    sConfigOC.OCMode = TIM_OCMODE_PWM1;
    sConfigOC.Pulse = 2000;
    sConfigOC.OCPolarity = TIM_OCPOLARITY_HIGH;
    sConfigOC.OCFastMode = TIM_OCFAST_DISABLE;
    if (HAL_TIM_PWM_ConfigChannel(&htim2, &sConfigOC, TIM_CHANNEL_1) != HAL_OK)
    {
    Error_Handler();
    }
    if (HAL_TIM_PWM_ConfigChannel(&htim2, &sConfigOC, TIM_CHANNEL_2) != HAL_OK)
    {
    Error_Handler();
    }
    /* USER CODE BEGIN TIM2_Init 2 */

    /* USER CODE END TIM2_Init 2 */
    HAL_TIM_MspPostInit(&htim2);

}

/**
    * @brief TIM3 Initialization Function
    * @param None
    * @retval None
    */
static void MX_TIM3_Init(void)
{

    /* USER CODE BEGIN TIM3_Init 0 */

    /* USER CODE END TIM3_Init 0 */

    TIM_ClockConfigTypeDef sClockSourceConfig = {0};
    TIM_MasterConfigTypeDef sMasterConfig = {0};
    TIM_OC_InitTypeDef sConfigOC = {0};

    /* USER CODE BEGIN TIM3_Init 1 */

    /* USER CODE END TIM3_Init 1 */
    htim3.Instance = TIM3;
    htim3.Init.Prescaler = 107;
    htim3.Init.CounterMode = TIM_COUNTERMODE_UP;
    htim3.Init.Period = 2082;
    htim3.Init.ClockDivision = TIM_CLOCKDIVISION_DIV1;
    htim3.Init.AutoReloadPreload = TIM_AUTORELOAD_PRELOAD_DISABLE;
    if (HAL_TIM_Base_Init(&htim3) != HAL_OK)
    {
    Error_Handler();
    }
    sClockSourceConfig.ClockSource = TIM_CLOCKSOURCE_INTERNAL;
    if (HAL_TIM_ConfigClockSource(&htim3, &sClockSourceConfig) != HAL_OK)
    {
    Error_Handler();
    }
    if (HAL_TIM_PWM_Init(&htim3) != HAL_OK)
    {
    Error_Handler();
    }
    sMasterConfig.MasterOutputTrigger = TIM_TRGO_RESET;
    sMasterConfig.MasterSlaveMode = TIM_MASTERSLAVEMODE_DISABLE;
    if (HAL_TIMEx_MasterConfigSynchronization(&htim3, &sMasterConfig) != HAL_OK)
    {
    Error_Handler();
    }
    sConfigOC.OCMode = TIM_OCMODE_PWM1;
    sConfigOC.Pulse = 2000;
    sConfigOC.OCPolarity = TIM_OCPOLARITY_HIGH;
    sConfigOC.OCFastMode = TIM_OCFAST_DISABLE;
    if (HAL_TIM_PWM_ConfigChannel(&htim3, &sConfigOC, TIM_CHANNEL_1) != HAL_OK)
    {
    Error_Handler();
    }
    /* USER CODE BEGIN TIM3_Init 2 */

    /* USER CODE END TIM3_Init 2 */
    HAL_TIM_MspPostInit(&htim3);

}

/**
    * @brief TIM4 Initialization Function
    * @param None
    * @retval None
    */
static void MX_TIM4_Init(void)
{

    /* USER CODE BEGIN TIM4_Init 0 */

    /* USER CODE END TIM4_Init 0 */

    TIM_ClockConfigTypeDef sClockSourceConfig = {0};
    TIM_MasterConfigTypeDef sMasterConfig = {0};
    TIM_OC_InitTypeDef sConfigOC = {0};

    /* USER CODE BEGIN TIM4_Init 1 */

    /* USER CODE END TIM4_Init 1 */
    htim4.Instance = TIM4;
    htim4.Init.Prescaler = 107;
    htim4.Init.CounterMode = TIM_COUNTERMODE_UP;
    htim4.Init.Period = 2082;
    htim4.Init.ClockDivision = TIM_CLOCKDIVISION_DIV1;
    htim4.Init.AutoReloadPreload = TIM_AUTORELOAD_PRELOAD_DISABLE;
    if (HAL_TIM_Base_Init(&htim4) != HAL_OK)
    {
    Error_Handler();
    }
    sClockSourceConfig.ClockSource = TIM_CLOCKSOURCE_INTERNAL;
    if (HAL_TIM_ConfigClockSource(&htim4, &sClockSourceConfig) != HAL_OK)
    {
    Error_Handler();
    }
    if (HAL_TIM_PWM_Init(&htim4) != HAL_OK)
    {
    Error_Handler();
    }
    sMasterConfig.MasterOutputTrigger = TIM_TRGO_RESET;
    sMasterConfig.MasterSlaveMode = TIM_MASTERSLAVEMODE_DISABLE;
    if (HAL_TIMEx_MasterConfigSynchronization(&htim4, &sMasterConfig) != HAL_OK)
    {
    Error_Handler();
    }
    sConfigOC.OCMode = TIM_OCMODE_PWM1;
    sConfigOC.Pulse = 1000;
    sConfigOC.OCPolarity = TIM_OCPOLARITY_HIGH;
    sConfigOC.OCFastMode = TIM_OCFAST_DISABLE;
    if (HAL_TIM_PWM_ConfigChannel(&htim4, &sConfigOC, TIM_CHANNEL_1) != HAL_OK)
    {
    Error_Handler();
    }
    /* USER CODE BEGIN TIM4_Init 2 */

    /* USER CODE END TIM4_Init 2 */
    HAL_TIM_MspPostInit(&htim4);

}

/**
    * @brief USART3 Initialization Function
    * @param None
    * @retval None
    */
static void MX_USART3_UART_Init(void)
{

    /* USER CODE BEGIN USART3_Init 0 */

    /* USER CODE END USART3_Init 0 */

    /* USER CODE BEGIN USART3_Init 1 */

    /* USER CODE END USART3_Init 1 */
    huart3.Instance = USART3;
    huart3.Init.BaudRate = 9600;
    huart3.Init.WordLength = UART_WORDLENGTH_8B;
    huart3.Init.StopBits = UART_STOPBITS_1;
    huart3.Init.Parity = UART_PARITY_NONE;
    huart3.Init.Mode = UART_MODE_TX_RX;
    huart3.Init.HwFlowCtl = UART_HWCONTROL_NONE;
    huart3.Init.OverSampling = UART_OVERSAMPLING_16;
    huart3.Init.OneBitSampling = UART_ONE_BIT_SAMPLE_DISABLE;
    huart3.AdvancedInit.AdvFeatureInit = UART_ADVFEATURE_NO_INIT;
    if (HAL_UART_Init(&huart3) != HAL_OK)
    {
    Error_Handler();
    }
    /* USER CODE BEGIN USART3_Init 2 */

    /* USER CODE END USART3_Init 2 */

}

/**
    * Enable DMA controller clock
    */
static void MX_DMA_Init(void)
{

    /* DMA controller clock enable */
    __HAL_RCC_DMA1_CLK_ENABLE();

    /* DMA interrupt init */
    /* DMA1_Stream0_IRQn interrupt configuration */
    HAL_NVIC_SetPriority(DMA1_Stream0_IRQn, 0, 0);
    HAL_NVIC_EnableIRQ(DMA1_Stream0_IRQn);
    /* DMA1_Stream4_IRQn interrupt configuration */
    HAL_NVIC_SetPriority(DMA1_Stream4_IRQn, 0, 0);
    HAL_NVIC_EnableIRQ(DMA1_Stream4_IRQn);
    /* DMA1_Stream5_IRQn interrupt configuration */
    HAL_NVIC_SetPriority(DMA1_Stream5_IRQn, 0, 0);
    HAL_NVIC_EnableIRQ(DMA1_Stream5_IRQn);
    /* DMA1_Stream6_IRQn interrupt configuration */
    HAL_NVIC_SetPriority(DMA1_Stream6_IRQn, 0, 0);
    HAL_NVIC_EnableIRQ(DMA1_Stream6_IRQn);

}

/**
    * @brief GPIO Initialization Function
    * @param None
    * @retval None
    */
static void MX_GPIO_Init(void)
{
    GPIO_InitTypeDef GPIO_InitStruct = {0};
/* USER CODE BEGIN MX_GPIO_Init_1 */
/* USER CODE END MX_GPIO_Init_1 */

    /* GPIO Ports Clock Enable */
    __HAL_RCC_GPIOH_CLK_ENABLE();
    __HAL_RCC_GPIOA_CLK_ENABLE();
    __HAL_RCC_GPIOC_CLK_ENABLE();
    __HAL_RCC_GPIOB_CLK_ENABLE();

    /*Configure GPIO pin Output Level */
    HAL_GPIO_WritePin(GPIOB, GPIO_PIN_2, GPIO_PIN_RESET);

    /*Configure GPIO pin : PC4 */
    GPIO_InitStruct.Pin = GPIO_PIN_4;
    GPIO_InitStruct.Mode = GPIO_MODE_IT_RISING;
    GPIO_InitStruct.Pull = GPIO_NOPULL;
    HAL_GPIO_Init(GPIOC, &GPIO_InitStruct);

    /*Configure GPIO pin : PB2 */
    GPIO_InitStruct.Pin = GPIO_PIN_2;
    GPIO_InitStruct.Mode = GPIO_MODE_OUTPUT_PP;
    GPIO_InitStruct.Pull = GPIO_NOPULL;
    GPIO_InitStruct.Speed = GPIO_SPEED_FREQ_VERY_HIGH;
    HAL_GPIO_Init(GPIOB, &GPIO_InitStruct);

    /* EXTI interrupt init*/
    HAL_NVIC_SetPriority(EXTI4_IRQn, 0, 0);
    HAL_NVIC_EnableIRQ(EXTI4_IRQn);

/* USER CODE BEGIN MX_GPIO_Init_2 */
/* USER CODE END MX_GPIO_Init_2 */
}

/* USER CODE BEGIN 4 */
uint8_t ParseCommand(const char* buf, uint32_t Len) {
    char cmd[64] = {0};
    if (Len >= sizeof(cmd)) Len = sizeof(cmd)-1;
    memcpy(cmd, buf, Len);

    for (int i = 0; i < Len; i++) {
        if (cmd[i]=='\r' || cmd[i]=='\n' || cmd[i]==';') {
            cmd[i] = '\0';
            break;
        }
    }
    if (strncmp(cmd, "SP", 2) == 0) {
        float x, y, z;
        if (sscanf(cmd + 2, "%f,%f,%f", &x, &y, &z) == 3) {
            memcpy(prev_pos, pos, 3*sizeof( float ) );
            pos[0] = x;
            pos[1] = y;
            pos[2] = z;

            HAL_UART_Transmit(&huart3, (uint8_t*)"POS_SET\r\n", 9, HAL_MAX_DELAY);
        } else {
            HAL_UART_Transmit(&huart3, (uint8_t*)"POS_ERR\r\n", 10, HAL_MAX_DELAY);
        }
        return 0;
    }// ECHO
    if (strcmp(cmd, "STOP") == 0) {
        requestedState = DRONE_STATE_EMERGENCY_STOP;
        HAL_UART_Transmit(&huart3, (uint8_t*)"EMERGENCY STOP REQUESTED\r\n", 26, HAL_MAX_DELAY);
        return 0;
    }

    if (strcmp(cmd, "LAND") == 0) {
        requestedState = DRONE_STATE_LANDING;
        HAL_UART_Transmit(&huart3, (uint8_t*)"LANDING REQUESTED\r\n", 19, HAL_MAX_DELAY);
        return 0;
    }
    if (strcmp(cmd, "ECHO") == 0) {
        HAL_UART_Transmit(&huart3, (uint8_t*)"ECHO\r\n", 6, HAL_MAX_DELAY);
        return 0;
    }
    // ARM
    if (strcmp(cmd, "ARM") == 0) {
        ArmESCs();
        HAL_UART_Transmit(&huart3, (uint8_t*)"ARMED\r\n", 7, HAL_MAX_DELAY);
        return 0;
    }
    // DISARM
    if (strcmp(cmd, "DISARM") == 0) {
        HAL_UART_Transmit(&huart3, (uint8_t*)"DISARMED\r\n", 10, HAL_MAX_DELAY);
        return 0;
    }
    // TEST-M
    if (strcmp(cmd, "TEST-M") == 0) {
        requested_speed=1220;
        requestedState = DRONE_STATE_TEST_CYCLE;
        HAL_UART_Transmit(&huart3, (uint8_t*)"TEST CYCLE\r\n", 12, HAL_MAX_DELAY);
        return 0;
    }
    if (strncmp(cmd, "ST", 2) == 0) {
        sscanf(cmd + 2, "%d", &requested_speed);
        requestedState = DRONE_STATE_TEST_CYCLE;
        char out[32];
        int n = snprintf(out, sizeof(out),"START in %d\r\n", requested_speed);
        HAL_UART_Transmit(&huart3, (uint8_t*)out, n, HAL_MAX_DELAY);
        return 0;
    }
    // CALB
    if (strcmp(cmd, "CALB") == 0) {
        ICM_CalibrateGyro(bias);
        ICM_ResetOrientation();
        HAL_UART_Transmit(&huart3, (uint8_t*)"Reset Successfully\r\n", 20, HAL_MAX_DELAY);
        return 0;
    }
    // GET_ACC-POS
    if (strcmp(cmd, "GET_ACC-POS") == 0) {
        ICM_GetEulerAnglesDeg(angles_deg);
        char out[128];
        int n = snprintf(out, sizeof(out),
            "ANG: %.2f,%.2f,%.2f; POS: %.2f,%.2f,%.2f\r\n",
            angles_deg[0], angles_deg[1], angles_deg[2], pos[0], pos[1], pos[2]);
        HAL_UART_Transmit(&huart3, (uint8_t*)out, n, HAL_MAX_DELAY);
        return 0;
    }
    if (strncmp(cmd, "LQR", 3) == 0){
        requestedState = DRONE_STATE_LQR;
        float x, y, z;
        if (sscanf(cmd + 3, "%f,%f,%f", &x, &y, &z) == 3) {
            memcpy(prev_pos, pos, 3*sizeof( float ) );
            pos_req[0] = x;
            pos_req[1] = y;
            pos_req[2] = z;

            //HAL_UART_Transmit(&huart3, (uint8_t*)"POS_SET\r\n", 9, HAL_MAX_DELAY);
        }
        //sscanf(cmd + 3, "%f,%f,%f", &pos_req[0],&pos_req[1],pos_req[2]);
        char out[64];
        int n = snprintf(out, sizeof(out),"POS REQ: %.2f,%.2f,%.2f\r\n", pos_req[0], pos_req[1], pos_req[2]);
        HAL_UART_Transmit(&huart3, (uint8_t*)out, n, HAL_MAX_DELAY);
        //HAL_Delay(4000);
        return 0;
    }
    // GET_GYRO
//    if (strcmp(cmd, "GET_GYRO") == 0) {
//        char out[64];
//        int n = snprintf(out, sizeof(out),
//            "GYRO: %.2f,%.2f,%.2f\r\n",
//            gyro[0], gyro[1], gyro[2]);
//        HAL_UART_Transmit(&huart3, (uint8_t*)out, n, HAL_MAX_DELAY);
//        return 0;
//    }
    if (strcmp(cmd, "GET_POS") == 0) {
        char out[128];
        int n = snprintf(out, sizeof(out),
            "POS: %.2f,%.2f,%.2f; PREV: %.2f,%.2f,%.2f\r\n",
            pos[0], pos[1], pos[2], prev_pos[0], prev_pos[1], prev_pos[2]);
        HAL_UART_Transmit(&huart3, (uint8_t*)out, n, HAL_MAX_DELAY);
        return 0;
    }
    // SET_COEFFS:k1,k2,...,k12
    if (strncmp(cmd, "SC", 2) == 0) {
        char out[64];
        int n = snprintf(out, sizeof(out), "%.1f,%.1f,%.1f,%.1f,%.1f,%.1f,%.1f,%.1f,%.1f,%.1f,%.1f,%.1f\r\n", k1,k2,k3,k4,k5,k6,k7,k8,k9,k10,k11,k12);
        HAL_UART_Transmit(&huart3, (uint8_t*)out, n, HAL_MAX_DELAY);
        return 0;
    }

    if (strncmp(cmd, "SK", 2) == 0) {
        int n =0;
        float k=0.0;
        if (sscanf(cmd + 2, "%d,%f", &n, &k) == 2) {
            char out[64];
            switch (n){
            case 1:
                k1=k;
                break;
            case 2:
                k2=k;
                break;
            case 3:
                k3=k;
                break;
            case 4:
                k4=k;
                break;
            case 5:
                k5=k;
                break;
            case 6:
                k6=k;
                break;
            case 7:
                k7=k;
                break;
            case 8:
                k8=k;
                break;
            case 9:
                k9=k;
                break;
            case 10:
                k10=k;
                break;
            case 11:
                k11=k;
                break;
            case 12:
                k12=k;
                break;
            }
            int l = snprintf(out, sizeof(out),"K%d SET TO %.2f\r\n", n,k);
            HAL_UART_Transmit(&huart3, (uint8_t*)out, l, HAL_MAX_DELAY);
        } else {
            HAL_UART_Transmit(&huart3, (uint8_t*)buf, Len, HAL_MAX_DELAY);
        }
        return 0;
        }
    HAL_UART_Transmit(&huart3, (uint8_t*)buf, Len, HAL_MAX_DELAY);
    return 1;
}

void HAL_GPIO_EXTI_Callback(uint16_t GPIO_Pin) {
}

uint8_t copy_rx;
void HAL_UART_RxCpltCallback(UART_HandleTypeDef *huart) {
    cmd_buf[cmd_idx++] = rx_byte;
    if (rx_byte == ';' || rx_byte == '\n' || cmd_idx >= RX_BUF_SIZE-1) {
        cmd_buf[cmd_idx] = '\0';
        ParseCommand(cmd_buf, cmd_idx);
        cmd_idx = 0;
    }
    HAL_UART_Receive_IT(&huart3, &rx_byte, 1);
}

void USB_CDC_RxHandler(uint8_t* Buf, uint32_t Len)
{
    HAL_UART_Transmit(&huart3, (uint8_t*)Buf, Len, 10);
//	if (ParseCommand((const char*)Buf, Len) == 1){
//		//CDC_Transmit_FS(Buf, Len);
//		HAL_UART_Transmit(&huart3, (uint8_t*)Buf, Len, 10);
//	}
}
/* USER CODE END 4 */

    /* MPU Configuration */

void MPU_Config(void)
{
    MPU_Region_InitTypeDef MPU_InitStruct = {0};

    /* Disables the MPU */
    HAL_MPU_Disable();

    /** Initializes and configures the Region and the memory to be protected
    */
    MPU_InitStruct.Enable = MPU_REGION_ENABLE;
    MPU_InitStruct.Number = MPU_REGION_NUMBER0;
    MPU_InitStruct.BaseAddress = 0x0;
    MPU_InitStruct.Size = MPU_REGION_SIZE_4GB;
    MPU_InitStruct.SubRegionDisable = 0x87;
    MPU_InitStruct.TypeExtField = MPU_TEX_LEVEL0;
    MPU_InitStruct.AccessPermission = MPU_REGION_NO_ACCESS;
    MPU_InitStruct.DisableExec = MPU_INSTRUCTION_ACCESS_DISABLE;
    MPU_InitStruct.IsShareable = MPU_ACCESS_SHAREABLE;
    MPU_InitStruct.IsCacheable = MPU_ACCESS_NOT_CACHEABLE;
    MPU_InitStruct.IsBufferable = MPU_ACCESS_NOT_BUFFERABLE;

    HAL_MPU_ConfigRegion(&MPU_InitStruct);
    /* Enables the MPU */
    HAL_MPU_Enable(MPU_PRIVILEGED_DEFAULT);

}

/**
    * @brief  This function is executed in case of error occurrence.
    * @retval None
    */
void Error_Handler(void)
{
    /* USER CODE BEGIN Error_Handler_Debug */
    /* User can add his own implementation to report the HAL error return state */
    __disable_irq();
    while (1)
    {
    }
    /* USER CODE END Error_Handler_Debug */
}

#ifdef  USE_FULL_ASSERT
/**
    * @brief  Reports the name of the source file and the source line number
    *         where the assert_param error has occurred.
    * @param  file: pointer to the source file name
    * @param  line: assert_param error line source number
    * @retval None
    */
void assert_failed(uint8_t *file, uint32_t line)
{
    /* USER CODE BEGIN 6 */
    /* User can add his own implementation to report the file name and line number,
        ex: printf("Wrong parameters value: file %s on line %d\r\n", file, line) */
    /* USER CODE END 6 */
}
#endif /* USE_FULL_ASSERT */
\end{lstlisting}


\endinput