\chapter*{Заключение}
\addcontentsline{toc}{chapter}{Заключение}
\label{ch:chap1}

В рамках выполнения работы были достигнуты поставленные цели: 
разработаны алгоритмы управления квадрокоптером для слежения за траекторией, 
проведено их моделирование в среде MATLAB/Simulink, выполнено сравнение 
качества переходных процессов, наиболее эффективный алгоритм 
реализован на реальном квадрокоптере.

Проведённое моделирование системы управления дроном на траекториях `восьмёрка' и `куб' 
позволило оценить эффективность трёх регуляторов: классического LQR, LQR с 
линеаризацией обратной связью (LQR+FB) и нелинейного MPC. Наибольшую точность 
демонстрируют регуляторы LQR+FB и MPC. Например, для траектории «восьмёрка» 
RMSE положения по осям X и Y у MPC составили 0.463 м и 0.305 м соответственно, 
что близко к результатам LQR+FB (0.470 м и 0.324 м). При этом MPC обеспечил минимальную
 ошибку по высоте (0.501 м), что подтверждает его преимущество в условиях отсутствия возмущений.

Регулятор LQR+FB показал значительно более высокую скорость моделирования 
благодаря меньшей вычислительной сложности по сравнению с MPC. 
Но нужно отметить, что регулятор MPC обладает высокой расширяемостью и 
конфигурируемостью. Возможность 
учёта нелинейностей, ограничений на состояния и управления, 
а также адаптация к агрессивным маневрам открывают перспективы 
для дальнейшего улучшения точности. Однако его реализация 
требует значительных вычислительных ресурсов, что может ограничивать применение в реальном времени.

В связи с отмеченными недостатками MPC регулятора было решено использовать
LQR регулятор с линеаризацией обратной связью для экспериментов на реальном квадрокоптере.
В ходе разработки квадрокоптера было реализовано следующее: программное обеспечение для квадрокоптера на
базе микроконтроллера STM32F722;
система коммуникации с квадрокоптером посредством MQTT протокола и микроконтроллера ESP8266; 
панель управления для браузера; система локализации с использованием визуальных маркеров
apriltag. Было проведено тестирование регулятора для стабилизации квадрокоптера на двух стендах. 

Выбор регулятора зависит от условий задачи и располагаемых ресурсов, соответственно 
LQR с линеаризацией обратной связью
будет очень хорошим и эффективным регулятором для решения многих задач, 
но MPC будет лучшим выбором, если есть большие вычислительные мощности и временные ресурсы для 
настройки параметров и модификации регулятора. 

\endinput
