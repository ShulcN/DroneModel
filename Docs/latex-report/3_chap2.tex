\chapter{Обзор математической модели квадрокоптера}
% \addcontentsline{toc}{chapter}{}
\label{ch:chap2}

\section{Описание модели}

Квадрокоптер обладает шестью степенями свободы:

\begin{itemize}
    \item Линейные координаты $X, Y, Z$,
    \item Угол крена \(\phi\) соответствует углу вращения вокруг оси \(X\). 
    \item Угол тангажа \(\theta\) соответствует углу вращения вокруг оси \(Y\).
    \item  Угол рыскания \(\psi\) соответствует углу вращения вокруг оси \(Z\).
\end{itemize}

\begin{figure}[ht]
    \centering
    \includegraphics[width=0.8 \textwidth]{math-model.png}
    \caption{Схематичное изображение квадрокоптера}
\end{figure}

\section{Локальная и глобальная системы координат}

Для линейных координат рассмотрим переход системы координат от 
глобальной \(X,Y,Z\) к системе координат связанной с квадрокоптером \(X_B,Y_B,Z_B\).

Чтобы перейти к системе координат, связанной с центром масс квадрокоптера, необходимо 
использовать матрицу поворота или, как её ещё называют матрицу косинусов.

Эта матрица представляет собой перемноженные три матрицы, которые осуществляют 
поворот вокруг оси \(X\), \(Y\) и \(Z\) соответственно по углам эйлера: \(\phi\), \(\theta\) и \(\psi\).

Рассмотрим пример получения матрицы поворота вокруг оси \(X\). 
При повороте вокруг оси \(X\) координата \(x\) не меняется.
Рассмотрим проекцию на плоскость \(OYZ\), представленную на \hyperref[fig:point-rot]{рисунке \ref*{fig:point-rot-2}}.

\begin{figure}[ht]
    \centering
    \hspace*{\fill}%
    \begin{subfigure}[b]{0.49\textwidth}
        \includegraphics[width=1 \textwidth]{point-rotation.png}
        \caption{}
        \label{fig:point-rot-1}
    \end{subfigure}
    \begin{subfigure}[b]{0.49\textwidth}
        \includegraphics[width=0.8 \textwidth]{point-rotation-2.png}
        \caption{}
        \label{fig:point-rot-2}
    \end{subfigure}
    \caption{Поворот точки вокруг оси. На рисунке (а) поворот точки в трёхмерном пространстве.
    На рисунке (б) поворот точки в плоскости OYZ.}
    \label{fig:point-rot}
\end{figure}

В проекции \(OYZ\) перейдем к полярным координатам: расстояние от начала координат \(r=\sqrt{y_1+z_1}\), начальный угол будет
 \(\alpha_1=\arctan(\frac{y_1}{z_1})\).

После поворота на угол \(\phi\) получим новый угол \(\alpha_2=\alpha_1+\phi\), а расстояние от начала координат не изменилось.

\begin{figure}[ht]
    \centering
    \includegraphics[width=0.5 \textwidth]{point-rotation-3.png}
    \caption{Проекция на плоскость OYZ в полярных координатах}
    \label{fig:point-rot-3}
\end{figure}

Выразим новые координаты \(y_2, z_2\) через расстояние от начала координат и угол \(\alpha_2\).
Получим: \(y_2=r\cos(\alpha_2)\), \(z_2=r\sin(\alpha_2)\). Где \(\cos(\alpha_2)=\cos(\alpha_1+\phi)=\cos(\alpha_1)\cos(\phi)-\sin(\alpha_1)\sin(\phi)\) и \(\sin(\alpha_2)=\sin(\alpha_1)\cos(\phi)+\cos(\alpha_1)\sin(\phi)\).

В итоге получаем следующую систему для перехода при повороте на угол \(\phi\) вокруг оси \(X\):
\begin{equation}
\begin{cases}
    x_2 = x_1\\
    y_2 = r (\cos(\alpha_1)\cos(\phi)-\sin(\alpha_1)\sin(\phi))\\
    z_2 = r (\sin(\alpha_1)\cos(\phi)+\cos(\alpha_1)\sin(\phi))
\end{cases}
\end{equation}
Заменим \(\cos(\alpha_1)=\frac{y_1}{r}\) и \(\sin(\alpha_1)=\frac{z_1}{r}\):
\begin{equation}
\begin{cases}
    x_2 = x_1\\
    y_2 = r \left(\frac{y_1}{r}\cos(\phi)-\frac{z_1}{r}\sin(\phi)\right)\\
    z_2 = r \left(\frac{z_1}{r}\cos(\phi)+\frac{y_1}{r}\sin(\phi)\right)
\end{cases}
\end{equation}
Упростим выражения:
\begin{equation}
\begin{cases}
    x_2 = x_1\\
    y_2 =  y_1 \cos(\phi)-z_1\sin(\phi)\\
    z_2 = y_1\sin(\phi)+z_1 \cos(\phi)
\end{cases}
\end{equation}

В матричном виде:

\begin{equation}
\begin{bmatrix}
    x_2 \\ y_2 \\z_2
\end{bmatrix} = 
\begin{bmatrix}
    1 & 0 & 0 \\
    0 & \cos(\phi) & -\sin(\phi) \\
    0 & \sin(\phi) & \cos(\phi)
\end{bmatrix}
\cdot
\begin{bmatrix}
    x_1 \\ y_1 \\z_1
\end{bmatrix}
\end{equation}

Аналогичные матрицы поворота используются для поворота вокруг осей \(Y\) и \(Z\).

\begin{equation}
R_y(\theta) = \begin{bmatrix}
    \cos(\theta) & 0 & \sin(\theta) \\
    0 & 1 & 0 \\
    -\sin(\theta) & 0 & \cos(\theta)
\end{bmatrix} 
\end{equation}

\begin{equation}
R_z(\psi) = \begin{bmatrix}
    \cos(\psi) & -\sin(\psi) & 0 \\
    \sin(\psi) & \cos(\psi) & 0 \\
    0 & 0 & 1
\end{bmatrix}
\end{equation}


Матрица направляющих косинусов \(R\) получается путем перемножения матриц поворота вокруг осей \(Z, Y, X\):

\begin{equation}
R = R_z(\psi) \cdot R_y(\theta) \cdot R_x(\phi) =
\begin{bmatrix}
C_{\psi} C_{\theta} & C_{\psi} S_{\theta} S_{\phi} - S_{\psi} C_{\phi} & C_{\psi} S_{\theta} C_{\phi} + S_{\psi} S_{\phi} \\
S_{\psi} C_{\theta} & S_{\psi} S_{\theta} S_{\phi} + C_{\psi} C_{\phi} & S_{\psi} S_{\theta} C_{\phi} - C_{\psi} S_{\phi} \\
-S_{\theta} & C_{\theta} S_{\phi} & C_{\theta} C_{\phi} \\
\end{bmatrix},
\end{equation}

где \( C - \cos\), \(S - \sin \). Матрица \(R\) позволяет перейти от глобальной к локальной системы координат, а чтобы
осуществить обратное преобразование необходимо взять обратную матрицу \(R^{-1}\), но так как матрица \(R\) - 
ортогональная, то \(R^{-1}=R^T\).

\section{Поступательное и вращательное движение}

В глобальной системе координат вводятся векторы линейных и угловых скоростей.

\begin{equation}
v = 
\begin{bmatrix}
    v_x \\ v_y \\ v_z
\end{bmatrix}
\omega = 
\begin{bmatrix}
    \omega_x \\ \omega_y \\ \omega_z
\end{bmatrix},
\end{equation}
где \(v, \omega\) --- векторы линейных и угловых скоростей соответственно в локальной системе координат, а 
\(v_x, v_y, v_z\) --- проекции линейных скоростей на соответствующие оси, 
\( \omega_x, \omega_y, \omega_z\) --- угловые скорости вокруг соответствующих осей.

\subsection{Поступательное движение и силы сопротивления}

Поступательное движение квадрокоптера описывается с помощью уравнения Ньютона-Эйлера. 

\begin{equation} m \ddot{r} = F_{\sum}, \end{equation}

где \(m\) --- масса квадрокоптера, 
\(r\) --- вектор положения центра масс, 
\(F_{\sum}\) --- сумма сил, действующих на квадрокоптер.

Квадрокоптер движется за счет силы тяги, которая создается при помощи электродвигателей, расположенных по углам рамы. Силу тяги можно представить 
в следующем виде:

\begin{equation}
T = T_{1} + T_{2} + T_{3} + T_{4}
\end{equation}
\begin{equation}
T_{i} = k_i \omega_i^2,
\end{equation}

где \(T\) --- суммарная сила тяги, которая направлена вдоль оси \(OZ\), поэтому представляет собой вектор \(\begin{bmatrix}
    0 & 0 & T
\end{bmatrix}^T\). \(T_i\) --- сила тяги отдельного электродвигателя с пропеллером.
\(k\) --- коэффициент тяги, который можно определить экспериментально, из данных об электродвигателе и пропеллерах или 
рассчитать приблизительное значение по формуле, предложенной в \cite{Lysukho}  \(k_i=\frac{1}{2}\rho C_{\gamma} S_{prop_i} r_{prop_i}^2\),
где \(\rho\) --- плотность воздуха, \(C_{\gamma}\) --- коэффициент подъемной силы, \(S_{prop_i}\) --- площадь окружности,
которую описывает винт радиусом \(r_{prop_i}\).


Также на квадрокоптер
действует сила тяжести и аэродинамические силы. Аэродинамические силы включают в себя флаттер лопастей, индуктивное сопротивление 
и лобовое сопротивление. 

Флаттер лопастей - это явление, возникающее из-за прогиба лопастей пропеллера при движения
квадрокоптера вперед. Лопасти, движущиеся вперед, создают больше подъемной силы, чем лопасти,
движущиеся назад, что приводит к наклону подъемной силы и созданию силы сопротивления. В 
статье [1] представлено моделирование лопастного флаттера следующим образом:

\begin{equation}
F_{flap} = -T A_{flap} v,
\end{equation}

где \(A_{flap}\) --- матрица флаттера, равная:

\begin{equation}
A_{flap} = \begin{bmatrix}
    c_a & -c_b & 0 \\
    c_b & c_a & 0 \\
    0 & 0 & 0 \\
\end{bmatrix},
\end{equation}

где \(c_a\) и \(c_b\) --- коэффициенты флаттера. Однако коэффициент \(c_b\) связан с поперечным флаттером 
лопастей, поэтому им можно пренебречь, так как продольный флаттер намного больше, чем поперечный. 

Индуктивное сопротивление --- это сопротивление, вызванное вихрями на торцах пропеллеров. Вихри 
образуются перетеканием воздуха из области под лопастью в область над лопастью.

\begin{equation}
F_{ind} = -T A_{ind} v,
\end{equation}

где \(A_{ind}=diag(c_{dx}, c_{dy}, 0)\), \(c_{dx}, c_{dy}\) - коэффициенты индуктивного сопротивления. Однако в силу 
симметричности квадрокоптера можно принять, что \(c_{dx}=c_{dy}=c_{d}\)


Лобовое сопротивление --- это сопротивление, вызванное трением воздуха о поверхность квадрокоптера.
Лобовое сопротивление будем учитывать, используя стандартное выражение для расчета сопротивления
\begin{equation}
F_{drag} = - \frac{1}{2} \rho  C_{drag} S \| r \| r,
\end{equation}

\(C_{drag}\) --- коэффициент сопротивления, \(S\) --- характерная площадь поверхности квадрокоптера.
\(\| v \|= \sqrt{r_x^2 + r_y^2 + r_z^2}\)

Сумма всех сил действующих на квадрокоптер в глобальной системе координат с учетом силы тяжести, которая
действует вдоль оси \(OZ\), будет выглядеть следующим образом:

\begin{equation}
m \ddot{r} =
T R - m \begin{bmatrix} 0 \\ 0 \\ g \end{bmatrix} + F_{flap} + F_{ind} + F_{drag}
\end{equation}

\begin{equation}
m \ddot{r} =
T R - m \begin{bmatrix} 0 \\ 0 \\ g \end{bmatrix} - T R A_{ind} R_T \dot{r} - T R A_{flap} R_T \dot{r} -  \frac{1}{2} \rho C_{drag} S \sqrt{v_x^2 + v_y^2 + v_z^2} \dot{r}
\end{equation}

Из этого уравнения можно получить систему уравнений для выражения положения квадрокоптера в глобальной системе координат:

\begin{equation}
\begin{cases}
    \ddot{r}_x = \frac{T \cos\theta \cos\psi - T (R A_{ind} R^T \dot{r})_x - T (R A_{flap} R^T \dot{r})_x - \frac{1}{2} \rho C_{drag} S \|\dot{r}\| \dot{r}_x}{m} \\
    \ddot{r}_y = \frac{T \cos\theta \sin\psi - T (R A_{ind} R^T \dot{r})_y - T (R A_{flap} R^T \dot{r})_y - \frac{1}{2} \rho C_{drag} S \|\dot{r}\| \dot{r}_y}{m} \\
    \ddot{r}_z = \frac{-T \sin\theta - T (R A_{ind} R^T \dot{r})_z - T (R A_{flap} R^T \dot{r})_z - \frac{1}{2} \rho C_{drag} S \|\dot{r}\| \dot{r}_z}{m} - g
\end{cases}
\end{equation}

\subsection{Вращательное движение}

Для перевода угловых скоростей из глобальной системы координат в локальной  
нужно ввести матрицу \(W\). Для вывода этой матрицы последовательно рассмотрим поворот вдоль каждой оси.
Также, как и в случае с матрицей поворота используется стандартная последовательность вращений: сначала вокруг оси
\(Z\), затем вокруг оси \(Y\) и затем вокруг оси \(X\).

Пусть исходная система координат, совпадающая с глобальной будет \(X_0, Y_0, Z_0\), соответственно, 
после первого поворота на угол \(\psi\) система координат будет \(X_1, Y_1, Z_1\), после второго поворота
на угол \(\theta\) система координат будет \(X_2, Y_2, Z_2\) и после третьего поворота на угол \(\phi\)
система координат будет \(X_3, Y_3, Z_3\).

Локальной системой координат является \(X_3, Y_3, Z_3\), в которой угловые скорости будут производными по
соответствующим углам Эйлера, однако угловые скорости \(\dot{\theta}\) и \(\dot{\psi}\) необходимо привести к 
локальной системе координат, так как они выполняются в других системах координат: \(X_2, Y_2, Z_2\) и \(X_1, Y_1, Z_1\).
Чтобы получить угловую скорость вокруг оси \(Y_3\) нужно привести систему координат в состояние до поворота на угол \(\phi\),
соответственно умножить вектор угловой скорости на матрицу поворота \(R_x(\phi)^{-1}=R_x(-\phi)\), аналогично для получения
угловой скорости вокруг оси \(Z_3\) необходимо также привести систему координат до поворота на угол \(\theta\) умножив на матрицу \(R_y(-\theta)\).

Угловая скорость \(\omega\) в локальной системе по компонентам:

\begin{equation}
\omega_{\phi}=
\begin{bmatrix}
    \dot{\phi} \\
    0 \\
    0
\end{bmatrix}
\end{equation}

\begin{equation}
\omega_{\theta}=
R_x({-\phi}) \begin{bmatrix}
    0 \\
    \dot{\theta} \\
    0
    \end{bmatrix}=
    \dot{\theta}
    \begin{bmatrix}
        0 \\
        \cos \phi \\
        -\sin \phi
    \end{bmatrix}
\end{equation}

\begin{equation}
\omega_{\psi}=
R_x({-\phi}) R_y({-\theta}) \begin{bmatrix}
    0 \\
    0 \\
    \dot{\psi}
    \end{bmatrix}=
    \dot{\psi}
    \begin{bmatrix}
        -\sin \theta \\
        \sin \phi \cos \theta \\
        \cos \phi \cos \theta
    \end{bmatrix}
\end{equation}


Вектор угловой скорости в локальной системе координат \(\omega=\omega_x+\omega_y+\omega_z\):

\begin{equation}
\omega=
\begin{bmatrix}
    1 & 0 & -\sin\theta \\
    0 & \cos\phi & \sin\phi\cos\theta \\
    0 &  -\sin\phi & \cos\phi\cos\theta
\end{bmatrix}
\cdot
\begin{bmatrix}
    \dot{\phi} \\
    \dot{\theta} \\
    \dot{\psi}
\end{bmatrix}
\end{equation}


Для перевода из локальной в глобальную \(W^{-1}\)

\begin{equation}
W^{-1} = \begin{bmatrix}
1 & \tan\theta \sin\phi & \tan\theta \cos\phi \\
0 & \cos\phi & -\sin\phi \\
0 & \frac{\sin\phi}{\cos\theta} & \frac{\cos\phi}{\cos\theta}
\end{bmatrix}.
\end{equation}

\subsection{Крутящие моменты}

Маневрирование квадрокоптера осуществляется за счет создания крутящих моментов. 
Крутящие моменты можно вычислить следующим образом:

\begin{equation}
\begin{cases}
    \tau_{\phi} = \tau_{q\phi} + \tau_{m\phi} + \tau_{p\phi}\\
    \tau_{\theta} = \tau_{q\theta} + \tau_{m\theta} + \tau_{p\theta}\\
    \tau_{\psi} = \tau_{q\psi},
\end{cases}
\end{equation}

где \(\tau_{q\phi}, \tau_{q\theta}, \tau_{q\psi}\) - моменты, создаваемые разницей сил тяги, которые создаются парами винтов.
Момент \(\tau_{m\phi}, \tau_{m\theta}\) - гироскопические моменты двигателей, \(\tau_{p\phi}, \tau_{p\theta}\)
 - гироскопические моменты винтов. Эти моменты могут быть выражены следующим образом:

\begin{equation}
\tau_{q\phi} = (T_3 - T_1)l
\end{equation}
\begin{equation}
\tau_{q\theta} = (T_2 - T_4)l 
\end{equation} 
\begin{equation}
\tau_{q\psi} = (T_2 + T_4 - T_1 - T_3)
\end{equation} 
\begin{equation}
\tau_{m\phi} = I_m \omega_{\theta}(\omega_1+\omega_3-\omega_2-\omega_4)
\end{equation} 
\begin{equation}
\tau_{m\theta} = I_m \omega_{\phi}(\omega_2+\omega_4-\omega_1-\omega_3)
\end{equation} 
\begin{equation}
\tau_{p\phi} = I_p \omega_{\theta}(\omega_1+\omega_3-\omega_2-\omega_4)
\end{equation} 
\begin{equation}
\tau_{p\theta} = I_p \omega_{\phi}(\omega_2+\omega_4-\omega_1-\omega_3)
\end{equation} 

Однако значения гироскопических моментов мало, поэтому ими можно пренебречь.
Для вращательного движения второй закон Ньютона будет выглядеть следующим образом:

\begin{equation}
I\dot{\omega}+\omega \times (I \omega) = \tau,
\end{equation}

где \(I=diag(I_{x}, I_{y}, I_{z})\) - это матрица инерции квадрокоптера, \(\omega \times (I \omega)\) - описывает гироскопические силы, которые возникают при вращении квадрокоптера.

\begin{equation}
\begin{cases}
    \tau_{\phi} =  I_{x} \dot{\omega}_{\phi} + (I_z-I_y) \omega_{\theta} \omega_{\psi} \\
    \tau_{\theta} = I_{y} \dot{\omega}_{\theta} + (I_x-I_z) \omega_{\phi} \omega_{\psi} \\
    \tau_{\psi} = I_{z} \dot{\omega}_{\psi} + (I_y-I_x) \omega_{\phi} \omega_{\theta} 
\end{cases}
\end{equation}

Из этой системы уравнений можно выразить угловые ускорения в локальной системе координат:

\begin{equation}
\begin{cases}
    \dot{\omega}_{\phi} = \frac{\tau_{\phi}+(I_z-I_y) \omega_{\theta} \omega_{\psi}}{I_{x}} \\
    \dot{\omega}_{\theta} = \frac{\tau_{\theta}+(I_x-I_z) \omega_{\phi} \omega_{\psi}}{I_{y}} \\
    \dot{\omega}_{\psi} = \frac{\tau_{\psi}+(I_y-I_x) \omega_{\phi} \omega_{\theta}}{I_{z}}
\end{cases}
\end{equation}

\section{Итоговая система уравнений динамики квадрокоптера}

Таким образом, итоговая система уравнений динамики квадрокоптера будет выглядеть следующим образом:

\begin{equation}
\begin{cases}
    \ddot{r}_x = \frac{T \cos\theta \cos\psi - T (R A_{ind} R^T \dot{r})_x - T (R A_{flap} R^T \dot{r})_x - \frac{1}{2} \rho C_{drag} S \|\dot{r}\| \dot{r}_x}{m} \\
    \ddot{r}_y = \frac{T \cos\theta \sin\psi - T (R A_{ind} R^T \dot{r})_y - T (R A_{flap} R^T \dot{r})_y - \frac{1}{2} \rho C_{drag} S \|\dot{r}\| \dot{r}_y}{m} \\
    \ddot{r}_z = \frac{-T \sin\theta - T (R A_{ind} R^T \dot{r})_z - T (R A_{flap} R^T \dot{r})_z - \frac{1}{2} \rho C_{drag} S \|\dot{r}\| \dot{r}_z}{m} - g \\
    \dot{\phi} = \omega_{\phi} + \tan\theta (\omega_{\theta} \sin\phi + \omega_{\psi} \cos\phi) \\
    \dot{\theta} = \omega_{\theta} \cos\phi - \omega_{\psi} \sin\phi \\
    \dot{\psi} = \frac{\sin\phi}{\cos\theta} \omega_{\theta} + \frac{\cos\phi}{\cos\theta} \omega_{\psi} \\
    \dot{\omega}_{\phi} = \frac{\tau_{\phi}+(I_z-I_y) \omega_{\theta} \omega_{\psi}}{I_{x}} \\
    \dot{\omega}_{\theta} = \frac{\tau_{\theta}+(I_x-I_z) \omega_{\phi} \omega_{\psi}}{I_{y}} \\
    \dot{\omega}_{\psi} = \frac{\tau_{\psi}+(I_y-I_x) \omega_{\phi} \omega_{\theta}}{I_{z}} \\
\end{cases}
\label{eq:system}
\end{equation}

В данной модели представлены динамические уравнения описывающие поступательное и вращательное 
движение квадрокоптера. Управление квадрокоптером осуществляется за счет изменения скорости 
вращения четырех электродвигателей, которые создают тягу. Соответственно, меняя соотношения скоростей
вращения электродвигателей, можно менять положение квадрокоптера в пространстве по 6 степеням свободы. Однако
для корректного управления с качественно хорошими переходными процессами необходимо синтезировать 
сложные системы управления.

\endinput