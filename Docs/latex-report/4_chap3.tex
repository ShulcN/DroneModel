\chapter{Анализ алгоритмов управления квадрокоптерами}
\label{ch:chap1}

В предыдущей главе была выведена математическая модель квадрокоптера. 
В данной модели представлены 
динамические уравнения описывающие поступательное и вращательное движение квадрокоптера.  
Управление квадрокоптером осуществляется за счет изменения скорости 
вращения четырех электродвигателей, которые создают тягу. 
Меняя соотношения скоростей
вращения электродвигателей, можно менять положение квадрокоптера в 
пространстве по шести степеням свободы. 

По решению задачи слежения за траекторией написано множество статей. Для управления квадрокоптером
используются классические, адаптивные и робастные регуляторы разных видов. Разные методы управления имеют свои достоинства 
и недостатки. 

\section{Классические методы управления}

Классические методы управления основаны на линейных моделях системы, что позволяет использовать
простые и эффективные регуляторы. Такие методы управления просты в реализации 
и не требуют больших вычислительных ресурсов. 


В качестве одного из базовых и эффективных алгоритмов управления можно назвать LQR регулятор по линеаризованной модели квадрокоптера.
Данная система управления простая в реализации, не требует тонких настроек и больших вычислительных ресурсов, однако
она не всегда может обеспечить необходимую точность управления, особенно в условиях сильных внешних воздействий. Также для реализации
классических методов управления необходимо линеаризовать систему, что может привести к потере точности управления в некоторых случаях.

Линеаризация системы производится, в основном двумя способами:
\begin{itemize}
    \item в точке равновесия;
    \item по методу обратной связи --- однако этот метод уже относится к нелинейному управлению.
\end{itemize}

Линеаризация в точке равновесия позволяет получить линейную модель системы, которая может быть использована для синтеза регуляторов.
Однако такая модель имеет недостаток — она не учитывает динамику системы в других точках пространства состояний, 
что ограничивает ее применение, например для агрессивных маневров. В основе метода линеаризации лежит приближение 
нелинейных уравнений системы с помощью разложения в ряд Тейлора.


Во многих статьях посвященных решению задачи слежения за траекторией для квадрокоптера - система управления разделятся на два контура: внутренний и внешний.
Внутренний контур отвечает за управление углом наклона квадрокоптера, а внешний контур отвечает за управление положением квадрокоптера в пространстве.
Таким образом, при сначала рассчитывается необходимое состояние по положению, а далее следующий регулятор определяет 
необходимые углы крена и тангажа. Положение по рысканию может включается как во внешний контур\cite{Linearization_two}, так и во внутренний\cite{LQR_feedback}.
Далее возможны различные комбинации алгоритмов управления для каждого из контуров.


\begin{figure}[ht]
    \centering
    \includegraphics[width=0.8 \textwidth]{inner-outer-1.png}
    \caption{Схема управления квадрокоптером с двумя контурами управления и линеаризацией по точке равновесия.}
    \label{fig:inner-outer-1}
\end{figure}


\section{Нелинейные методы управления}


В отличие от классических линейных методов управления, имеющих ограниченные возможности, 
системы управления, основанные на полной нелинейной модели квадрокоптера, обеспечивают более высокую точность и гибкость. 
Они способны адаптироваться к быстро меняющимся условиям окружающей среды и учитывать сложную динамику системы. 
Однако такие регуляторы сложнее в реализации и требуют значительно больших вычислительных ресурсов.

К числу популярных нелинейных методов управления относятся:

\begin{itemize}
    \item нелинейное управление с обратной связью (Feedback Linearization),
    \item пошаговое обратное управление (Backstepping),
    \item управление методом скользящего режима (Sliding Mode Control),
    \item управление на основе предсказательной модели (Model Predictive Control, MPC).
\end{itemize}


\subsection{Метод обратной связи (Feedback Linearization)}

Линеаризация по методу обратной связи позволяет получить более точную модель системы, которая учитывает динамику системы в
разных точках пространства состояний. Основная идея метода заключается в 
замене управления в системе на виртуальное, которое делает систему линейной, а далее виртуальное управление обратно преобразовывается в реальное.

\begin{figure}[ht]
    \centering
    \includegraphics[width=0.8 \textwidth]{inner-outer-3.png}
    \caption{Схема управления квадрокоптером с двумя контурами управления и линеаризацией по методу обратной связи.}
    \label{fig:inner-outer-2}
\end{figure}

В статье 
`Different linearization control techniques for a quadrotor system`
описаны алгоритмы управления квадрокоптером на основе LQR регулятора с линеаризацией сразу двумя способами —
в первом варианте используется линеаризация по точке равновесия, во втором — по методу обратной связи по схеме приведенной
на \hyperref[fig:point-rot]{рисунке \ref*{fig:inner-outer-2}}\cite{LQR_feedback}. 
Полученные результаты 
показали, что линеаризация методом обратной связи обеспечивает более высокую точность управления, преимущественно 
лучше справляется в задаче стабилизации квадрокоптера по углам крена и тангажа.

\subsection{Метод обратного пошагового управления (Backstepping)}

Backstepping — это подход к синтезу нелинейных регуляторов, 
который основан на пошаговом регулировании вложенных подсистем. 
Данный метод хорошо применим к системам с иерархической структурой, 
как в случае квадрокоптера, где положение зависит от ориентации, а ориентация — от управляющих моментов.

В этом методе так же, как и в рассмотренном ранее методе управления в системе заменяются на виртуальные, но 
это управления синтезируются последовательно и как правило делается с помощью кандидата функции Ляпунова, таким 
образом доказывается устойчивость каждого уровня управления.

Backstepping один из самых распространенных способов управления квадрокоптером, который комбинируется с различными модификациями.

В статье `Robust Backstepping Sliding Mode
Control for a Quadrotor Trajectory
Tracking Application` предложен двухконтурный регулятор для квадрокоптера, 
в котором используются Backstepping и интегральное скользящее управление \cite{RobustSlidingMode}. 
Внутренний контур по ориентации реализован как робастный backstepping контроллер на скользящих режимах, 
а внешний контур по позиции — с помощью интегральных скользящих режимов.
Основная сложность — в настройке большого числа параметров и учёте всех возмущений. 
Несмотря на это, система была устойчивой, даже при действии сильных помех.

\subsection{Метод скользящего режима}

Одним из широко используемых методов робастного управления в задачах управления квадрокоптером является 
метод скользящего режима (Sliding Mode Control). 
Данный метод обеспечивает устойчивость системы даже при наличии 
существенных внешних возмущений и неопределённостей в модели.

Суть метода заключается в проектировании скользящей поверхности в пространстве состояний системы. 
Управляющее воздействие формируется таким образом, чтобы система как можно быстрее вышла на эту поверхность 
и затем оставалась на ней, двигаясь к заданному положению равновесия. 
Таким образом система борется с разными возмущениями, оставаясь на скользящей поверхности.
Но этот метод нуждается в модификация и доработках, чтобы избежать частых переключений управления.

В статье `Robust adaptive nonsingular fast terminal sliding-
mode tracking control for an uncertain quadrotor UAV subjected to distur-
bances' представлен продвинутый подход к управлению квадрокоптером на
основе скользящих режимов\cite{AdaptiveSlidingMode}. В алгоритме есть несколько модификаций: адаптивная оценка верхних 
границ неопределенностей, терминальная модификация, которая позволяет системе быстро сходиться за конечное время. 
Система управления сложная, но результаты моделирования, приведенные в работе, впечатляют своей эффективностью. Также 
в работе приведено сравнение с подходами на основе Backstepping и Feedback Linearization.




\subsection{Model Predictive Control (MPC)}

Одним из самых универсальных и современных алгоритмов является MPC (Model Predictive Control), который
использует дискретную модель квадрокоптера для прогнозирования его поведения на некотором горизонте планирования. 
Управляющее воздействие определяется как решение задачи оптимизации.
Алгоритм способен учитывать внешние возмущения, а также может быть адаптирован для агрессивного маневрирования. 
MPC может быть легко модифицирован, например, можно ввести ограничения на отдельные компоненты вектора состояния и 
управления.
Основными недостатками MPC являются очень высокая вычислительная нагрузка и сложность реализации в режиме реального времени.
MPC регулятор также обладает большим количеством настраиваемых параметров, такие как --- 
веса стоимости управления, веса стоимости состояния, горизонт планирования, количество итераций для оптимизации, 
алгоритм оптимизации и другое. Такое обилие настраиваемых параметров с одной стороны позволяет
разработать наиболее подходящий регулятор для конкретных задач, а с другой стороны увеличивает сложность разработки.

В работе `Nonlinear Model Predictive Control for Unmanned
Aerial Vehicles' представлен сравнительный анализ линейного и нелинейного MPC-регуляторов для управления квадрокоптером с 
учётом ограничений на входы и состояния\cite{aerospace4020031}. Линейная модель получена путём линеаризации в окрестности точки зависания, 
а нелинейная модель построена в виде SDC-представления, 
где матрицы системы зависят от текущего состояния. Для оптимизации управления используется последовательный 
квадратичный программный решатель (SQP). 
По полученным результатам симуляций видно, что нелинейный MPC был гораздо эффективнее --- управление меньше, а 
точность лучше. Но нужно понимать, что регулятор по нелинейной модели требует гораздо больше вычислительной мощности.


\endinput