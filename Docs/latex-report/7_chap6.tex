\chapter{Реализация квадрокоптера на базе STM32F722}

Основной задачей являлась 
реализация эффективной системы стабилизации и управления 
полетом с использованием линейного квадратичного регулятора LQR 
с линеаризацией обратной связью. Выбор 
LQR был обусловлен его хорошим соотношением между вычислительной 
нагрузкой и качеством управления, что видно 
по результатам моделирования приведенным в предыдущей главе.

\section{Аппаратная часть квадрокоптера}

В результате разработан квадрокоптер, построенный на базе готового 
полетного контроллера Radiolink F722, в основе которого лежит 
микроконтроллер STM32F722\cite{RadiolinkF722}\cite{STM32F722}. 
Полетный контроллер Radiolink F722 включает в себя встроенный датчик пространственного положения 
и гироскопа ICM42688-P. Взаимодействие с датчиком осуществлялось по интерфейсу SPI\cite{ICM42688P}.


\begin{figure}[ht]
    \centering
    \includegraphics[width=0.3 \textwidth]{flight-controller-1.png}
    \caption{Полетный контроллер Radiolink F722}
    \label{}
\end{figure}

Управление четырьмя моторами осуществлялось с помощью широтно-импульсной модуляции
через электронные регуляторы скорости(ESC). На квадрокоптер были установлены популярные бесколлектрные моторы
EMAX ECO || 2306 1300 KV\cite{EmaxECO1300KV}. KV --- это стандартная характеристика бесколлекторных моторов, которое означает
количество оборотов на вольт без нагрузки.

\begin{figure}[ht]
    \centering
    \includegraphics[width=0.45 \textwidth]{motor-1.png}
    \caption{Мотор EMAX ECO || 2306 1300 KV}
    \label{}
\end{figure}


В качестве рамы квадрокоптера было выбрано готовое решение --- корпус Mark 4 295 мм.

\begin{figure}[ht]
	\centering
	\includegraphics[width=0.8 \textwidth]{drone.JPG}
	\caption{Квадрокоптер в собранном виде}
	\label{}
\end{figure}


В среде САПР среде Solidworks была реализована 3D модель квадрокоптера. Эта модель была
импортирована в среду Matlab/Simulink для имитационного моделирования.

\begin{figure}[ht]
	\centering
	\includegraphics[width=0.8 \textwidth]{cad-model-1.png}
	\caption{3D модель квадрокоптера в Solidworks}
	\label{}
\end{figure}

\section{Программное обеспечение квадрокоптера}

Разработка программного обеспечения велась в среде 
STM32CubeIDE. Предварительно были изучены исходные файлы открытой 
прошивки полетных контроллеров Betaflight, благодаря которым удалось произвести
обратное проектирование полетного контроллера для разработки собственного программного обеспечения\cite{Betaflight}.

Для удобной и универсальной коммуникации с квадрокоптером была выбрана коммуникация по 
протоколу MQTT, который часто используется в проектах интернет вещей в сетях 
с низкой пропускной способностью\cite{MQTT}. Таким образом сообщения доходят быстро и довольно стабильно
при должной настройке. Так как микроконтроллер STM32F722 не оснащен WI-FI модулем к нему 
был подключен микроконтроллер ESP8266, который обладает встроенным WI-FI модулем и способен 
принимать команды по MQTT, передавая их на STM32F722 по интерфейсу UART.

\begin{figure}[ht]
	\centering
\hspace*{\fill}%
	\begin{subfigure}[b]{0.49\textwidth}
        \centering
		\includegraphics[height=9cm,keepaspectratio]{esp8266.png}
		\caption{}
		\label{fig:tiger1}
	\end{subfigure}
\hfill
	\begin{subfigure}[b]{0.49\textwidth}
        \centering
		\includegraphics[height=9cm,keepaspectratio]{drone-inside-1.png}
        \caption{}
		\label{fig:tiger2}
	\end{subfigure}
\hspace*{\fill}%
	\caption{На рисунке (а) ESP8266. На рисунке (б)Встроенный ESP8266 на квадрокоптере}
	\label{fig:tiger}
\end{figure}

\newpage

Также был сделана панель управления для управления квадрокоптером через окно
браузера. 


\begin{figure}[ht]
	\centering
\hspace*{\fill}%
	\begin{subfigure}[b]{0.49\textwidth}
        \centering
		\includegraphics[height=9cm,keepaspectratio]{dashboard-1.png}
		\caption{}
		\label{fig:tiger1}
	\end{subfigure}
\hfill
	\begin{subfigure}[b]{0.49\textwidth}
        \centering
		\includegraphics[height=9cm,keepaspectratio]{dashboard-2.png}
        \caption{}
		\label{fig:tiger2}
	\end{subfigure}
\hspace*{\fill}%
	\caption{Панель управления квадрокоптером в браузере}
	\label{fig:tiger}
\end{figure}

В качестве брокера сообщений для обеспечения передачи команд от компьютера к 
микроконтроллеру ESP8266 был выбран популярный - Mosquitto, который можно развернуть 
локальной машине\cite{Mosquitto}.

\newpage

\section{Локализация квадрокоптера}

Для решения задачи локализации квадрокоптера было решено использовать визуальные маркеры apriltag\cite{AprilTag}.
В библиотеке OpenCV уже есть встроенные функции для детектирования и отслеживания маркеров apriltag\cite{OpenCV}\cite{Aruco}.
Были также написана логика для калибровки камеры для получения координат 
 по осям: X, Y, Z в системе СИ. Код приложения для отслеживания визуального маркера 
запускался на одноплатном компьютере Raspberry PI 4b с 4 Гб оперативной памяти. 
Использовалась камера Raspberry Pi Camera Module Rev 1.3 с разрешением 5 мегапикселей.
На каждом кадре осуществляется поиск маркера, рассчитанные координаты отправляются по
MQTT на квадрокоптер.


\begin{figure}[ht]
    \centering
    \includegraphics[width=0.5 \textwidth]{drone-apriltag.png}
    \caption{Apriltag 36h11 закрепленный на квадрокоптере}
    \label{}
\end{figure}

\begin{figure}[ht]
	\centering
\hspace*{\fill}%
	\begin{subfigure}[b]{0.49\textwidth}
        \centering
		\includegraphics[height=9cm,keepaspectratio]{camera+drone_1.png}
		\caption{}
		\label{fig:tiger1}
	\end{subfigure}
\hfill
	\begin{subfigure}[b]{0.49\textwidth}
        \centering
		\includegraphics[height=9cm,keepaspectratio]{camera+drone_2.png}
        \caption{}
		\label{fig:tiger2}
	\end{subfigure}
\hspace*{\fill}%
	\caption{Детектирование тэга на камере}
	\label{fig:tiger}
\end{figure}

\newpage


\section{Тестирование квадрокоптера}

Разработанный прототип частично тестировался в полевых условиях на двух стендах и в свободном полете.


\begin{figure}[ht]
    \centering
    \includegraphics[width=0.5 \textwidth]{drone_fly_1.png}
    \caption{Тестирование квадрокоптера в свободном полете}
    \label{}
\end{figure}

\begin{figure}[ht]
	\centering
\hspace*{\fill}%
	\begin{subfigure}[b]{0.49\textwidth}
        \centering
		\includegraphics[height=9cm,keepaspectratio]{test-2.png}
		\caption{}
		\label{fig:tiger1}
	\end{subfigure}
\hfill
	\begin{subfigure}[b]{0.49\textwidth}
        \centering
		\includegraphics[height=9cm,keepaspectratio]{test-1.png}
        \caption{}
		\label{fig:tiger2}
	\end{subfigure}
\hspace*{\fill}%
	\caption{Тестирование квадрокоптера на стендах}
	\label{fig:tiger}
\end{figure}

Тестирование на предмет слежения за траекторией не удалось в силу трудностей с построением специального полигона, 
однако
регулятор показал свою работоспособность в решении задачи стабилизации 
квадрокоптера. 


\endinput